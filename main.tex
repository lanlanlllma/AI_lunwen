\documentclass{cumcmthesis}
\usepackage[framemethod=TikZ]{mdframed}
\usepackage{url}   % 网页链接
\usepackage{subcaption} % 子标题
\usepackage{tikz}
\usetikzlibrary{positioning,arrows.meta,calc}
% Avoid hyperref warnings about tokens in PDF strings (from xeCJK/ctex internals)
\makeatletter
\pdfstringdefDisableCommands{%
    \def\leavevmode{}%
    \def\leavevmode@ifvmode{}%
    \def\kern#1{}%
}
\makeatother
% Improve PDF bookmark handling and encoding for hyperref
\usepackage{bookmark}
\hypersetup{unicode=true, pdfencoding=auto}

\title{智能制造场景零件的智能检测及无序分拣}

\date{\today}

\author{1,23}
 
\begin{document}
\maketitle
\newpage
% Minimal document body to avoid Emergency stop during compilation
\begin{abstract}
随着智能制造的发展,工件检测和缺陷识别在工业生产中变得越来越重要。传统的人工检测方法效率低且易受主观因素影响,难以满足现代制造业对高效、精准检测的需求。
本文首先提出了基于计算机视觉的工件检测方案,利用深度学习技术实现对工件表面缺陷的自动识别。其次,设计了一套无序分拣系统,通过图像处理和机器学习算法,实现对不同类型工件的分类和分拣。最后,通过实验验证了所提方案的有效性和实用性,结果表明该系统在提高检测效率和准确率方面具有显著优势。
\keywords{工件检测, 缺陷识别, 计算机视觉, 深度学习}
\end{abstract}

\section{项目背景与意义}

随着智能制造的快速发展,生产线正逐步向柔性化、无人化和高精度方向转型。传统制造业中,零部件的检测与分拣往往依赖人工操作,存在以下问题:

\begin{enumerate}
    \item \textbf{检测精度不稳定:} 人工检测容易受主观因素影响,对细微划痕、脏污等缺陷识别准确率低。
    \item \textbf{劳动强度高、效率低:} 大量重复性检测与分拣工作占用了人力资源,生产节拍难以匹配自动化产线节奏。
    \item \textbf{零件种类多、形态复杂:} 以“三钢柱塞泵连杆配件原厂件泵头零件”为例,其具有多种型号(22/26/30/60/80/90/100/120),不同尺寸与表面反光特性导致传统视觉检测系统泛化能力不足。
    \item \textbf{无序堆放带来挑战:} 在制造环节中,零件常处于随机堆放状态,传统机械臂需人工摆放或依赖固定夹具,柔性差。
\end{enumerate}

基于此,本项目以“\textbf{智能制造场景下的零件智能检测及无序分拣系统}”为研究对象,聚焦于三钢柱塞泵系列零件的自动化检测与抓取,构建一套从缺陷识别到自主分拣的智能化系统。

系统通过 \textbf{LWDetr} 轻量化目标检测模型实现表面缺陷(划痕、脏污)检测,再结合 \textbf{GraspNET} 无序抓取网络实现复杂堆叠环境下的智能分拣,最终完成零件的“自动识别—缺陷判定—自主抓取—分拣分类”全流程闭环。

该项目的研究与应用将:
提升制造业中零件检测与分拣环节的自动化水平,降低人工成本与误检率,推动智能制造向“感知—决策—执行”一体化方向发展,并为泵类零件等精密部件的质量控制提供智能化解决方案。

\section{场景与系统总体架构}

本项目以“三钢柱塞泵连杆配件原厂件泵头零件”的生产与检测环节为研究对象,构建了一个面向智能制造场景的零件智能检测与无序分拣系统。系统整体由 \textbf{感知层、决策层、执行层} 三部分构成,涵盖了从图像采集、缺陷识别、抓取规划到机械臂执行的完整流程。

\subsection{场景描述}

在实际生产车间中,待检测零件经过加工后被随机放置在料框中,表面可能存在划痕、油污或其他缺陷。传统的人工检测与分拣效率低下,且检测标准不稳定。  
因此,本系统部署在自动化工作站中,实现如下功能:

\begin{itemize}
    \item 通过工业相机采集不同型号泵头零件的表面图像;
    \item 利用 \textbf{LWDetr} 模型进行表面划痕、脏污缺陷检测;
    \item 结合 \textbf{GraspNET} 网络实现堆叠零件的抓取姿态预测;
    \item 由机械臂完成目标件的自动抓取与分类分拣;
    \item 系统将检测结果与分拣数据实时上传至生产管理系统,实现可追溯与统计。
\end{itemize}

\subsection{系统总体架构}

系统由图像采集模块、智能检测模块、抓取规划模块、机械臂控制模块及上位机监控模块组成。整体结构如图~\ref{fig:system_architecture} 所示。

\begin{figure}[h]
    \centering
    \begin{tikzpicture}[
        node distance=1.8cm,
    block/.style={rectangle, draw, rounded corners=3pt, text centered, minimum height=1cm, minimum width=3cm, inner sep=6pt, outer sep=0pt, fill=gray!10},
    line/.style={draw, -{Stealth[length=6pt,width=6pt]}, thick, shorten >=1.5pt, shorten <=1.5pt}
    ]

    % nodes
    \node[block] (camera) {工业相机与光源系统};
    \node[block, right=of camera, xshift=0cm] (detect) {LWDetr缺陷检测模块};
    \node[block, below=of detect] (grasp) {GraspNET无序抓取模块};
    \node[block, left=of grasp, xshift=-0cm] (arm) {机械臂执行系统};
    \node[block, below=of grasp] (upper) {上位机与数据管理模块};

    % lines (use node anchors so arrows hit node borders precisely)
    \path[line] (camera.east) -- (detect.west);
    \path[line] (detect.south) -- (grasp.north);
    \path[line] (grasp.west) -- (arm.east);
    \path[line] (arm.south) -- (upper.north);
    % dashed connections routed from node edges then around to upper.north
    % \path[line, dashed] (detect.east) -- ++(0,-1.2) -| (grasp.east);
    \path[line, dashed] (grasp.south) -- ++(0,-1.2) -| (upper.north);

    \end{tikzpicture}
    \caption{智能检测与无序分拣系统总体架构示意图}
    \label{fig:system_architecture}
\end{figure}

\subsection{软硬件组成}

\begin{itemize}
    \item \textbf{硬件部分:}
    \begin{itemize}
        \item 工业相机(2000万像素)+ 环形光源;
        \item 六自由度协作机械臂;
        \item GPU 边缘计算单元(NVIDIA Jetson AGX Orin / RTX 系列显卡);
        \item 工控机与 PLC 通讯模块;
        \item 输送带与料框定位装置。
    \end{itemize}

    \item \textbf{软件部分:}
    \begin{itemize}
        \item LWDetr 轻量级缺陷检测模型;
        \item GraspNET 三维抓取姿态预测网络;
        \item ROS2 通信与任务调度框架;
        \item PyTorch 深度学习推理模块;
        \item 可视化与数据管理界面(基于 Qt / Web 控制台)。
    \end{itemize}
\end{itemize}

通过上述架构设计,系统可实现对不同规格泵头零件的自动检测、定位与无序抓取,显著提高生产线的柔性与智能化水平。


\section{技术方案}

本系统采用“云-边协同”的智能制造体系结构,
结合本地轻量化模型与云端多模态大模型,
通过 \textbf{MCP(Model Context Protocol,模型上下文协议)} 实现检测、分拣、数据分析及人机交互的一体化智能优化。系统的核心算法框架包括三个层面:\textbf{边缘推理层}、\textbf{云端智能层}、以及\textbf{人机协同层}。

\subsection{边缘推理层:轻量化检测与抓取决策}

在边缘侧部署 \textbf{LWDetr} 与 \textbf{GraspNET} 模型,实现对现场数据的实时处理:

\begin{itemize}
    \item \textbf{LWDetr 模型:}  
    基于 Transformer 架构的轻量级目标检测网络,通过特征金字塔和自注意力机制,在保持高检测精度的同时显著降低计算量,可在 Jetson AGX Orin 等嵌入式平台上实现实时推理,用于检测泵头零件表面的划痕与脏污。

    \item \textbf{GraspNET 模型:}  
    基于点云输入的深度抓取预测网络,通过并行卷积与姿态回归模块,直接输出三维抓取位姿(位置 + 姿态 + 抓取置信度),实现复杂堆叠场景下的无序抓取。

    \item \textbf{本地优化:}  
    推理端集成 TensorRT 加速与 INT8 量化,以实现高吞吐率和低延迟,满足生产线节拍需求。
\end{itemize}

\subsection{云端智能层:多模态大模型与 MCP 服务}

云端部署多模态工业大模型,作为系统的智能中枢,通过 \textbf{MCP(Model Context Protocol} 与边缘节点进行数据交互与知识协同,主要功能包括:

\begin{itemize}
    \item \textbf{多模态数据理解:}  
    云端大模型可对边缘上传的图像、点云及日志信息进行联合分析,实现缺陷类型判别、异常样本聚类、以及检测模型自动再训练。

    \item \textbf{MCP 服务机制:}  
    MCP 提供标准化的任务通信接口(RESTful API / ROS2 Bridge),可动态下发检测参数、优化模型权重,并支持在线微调与模型版本管理。

    \item \textbf{知识增强决策:}  
    云端模型结合历史生产数据与知识图谱,对检测结果进行语义理解与趋势分析,从而对设备状态、产品质量给出预测性维护与调度优化建议。
\end{itemize}

\subsection{人机协同层:智能交互与多模态可视化}

通过云端大模型与 MCP 服务接口,系统具备自然语言与视觉多模态的人机交互能力:

\begin{itemize}
    \item 操作人员可通过语音或文本指令与系统交互,例如“展示今日检测缺陷分布”或“调整抓取策略”;
    \item 系统可通过多模态反馈(图像、语音、图表)解释检测与分拣过程,提升可解释性;
    \item 在异常情况下,系统自动生成事件报告并通过大模型生成可读性高的运维日志。
\end{itemize}

\subsection{系统数据流与云边协同示意}

如图~\ref{fig:cloud_edge_architecture} 所示,系统在检测、分析、反馈三个阶段实现边缘实时响应与云端智能决策的协同工作。

\begin{figure}[h]
    \centering
    \begin{tikzpicture}[
        node distance=2.2cm,
        block/.style={rectangle, draw, rounded corners, text centered, minimum height=1cm, minimum width=3cm, fill=gray!10},
        line/.style={draw, -{Stealth[length=6pt,width=6pt]}, thick, shorten >=1.5pt, shorten <=1.5pt}
    ]

    % nodes
    \node[block] (camera) {工业相机 / 点云采集};
    \node[block, right=of camera] (edge) {\shortstack{边缘计算节点\\(LWDetr + GraspNET)}};
    \node[block, below=of edge, yshift=0.15cm] (cloud) {\shortstack{云端多模态大模型\\+ MCP 协同服务}};
    \node[block, below=of camera] (ui) {人机交互终端};

    % lines
    \path[line] (camera.east) -- (edge.west) node[midway, above]{实时检测};
    \path[line] (edge.south) -- (cloud.north) node[midway, left]{数据上传 / 参数同步};
    \path[line] (cloud.west) -- (ui.east) node[midway, below]{多模态交互};
    \path[line] (ui.north) --(camera.south) node[midway, left]{指令下发 / 调度控制};

    \end{tikzpicture}
    \caption{云-边协同与多模态人机交互架构示意图}
    \label{fig:cloud_edge_architecture}
\end{figure}

通过上述架构,系统实现了:
\begin{itemize}
    \item 边缘侧的高效实时推理;
    \item 云端侧的知识增强与全局优化;
    \item 多模态交互的人机智能协同;
    \item 模型的自进化与持续优化。
\end{itemize}

\subsection{无序抓取模块设计}

在智能制造场景中,零件通常以无序堆叠方式放置在料框或传送带上,其姿态多变、部分遮挡严重,传统基于几何模板或规则匹配的方法难以满足实时性与鲁棒性要求。为此,本项目采用 \textbf{GraspNET} 深度学习抓取网络,实现无序环境下的三维抓取位姿预测与动作规划。

\subsubsection{网络结构与输入输出}

GraspNET 以点云数据为主要输入,核心结构包括:
\begin{itemize}
    \item \textbf{点云特征编码模块:} 采用 PointNet++ 或 SparseConv3D 提取空间局部特征,获得零件表面的几何描述;
    \item \textbf{候选抓取点生成模块:} 基于几何中心与表面法向分布,生成多组潜在抓取姿态;
    \item \textbf{抓取质量评估网络:} 通过卷积与注意力机制,预测每个候选姿态的抓取置信度(grasp quality score);
    \item \textbf{姿态回归模块:} 输出抓取位姿的六维参数 $\{x, y, z, \theta, \phi, \psi\}$,用于机械臂轨迹规划。
\end{itemize}

\subsubsection{点云采集与预处理}

\begin{itemize}
    \item 使用 RGB-D 相机或结构光传感器获取料框中零件的彩色与深度信息;
    \item 通过背景差分与 RANSAC 平面去除算法,剔除料框底面及噪声点;
    \item 采用体素下采样(Voxel Grid)与法线估计,生成高质量输入点云;
    \item 对检测出的缺陷区域附加语义标记,使抓取模块在规划时自动避开表面损伤部位。
\end{itemize}

\subsubsection{抓取推理与执行流程}

\begin{enumerate}
    \item 边缘端实时接收 \textbf{LWDetr} 的检测结果及位姿信息;
    \item 结合深度图生成对应的点云区域;
    \item 将点云输入 \textbf{GraspNET} 模型,获得抓取候选集与置信度;
    \item 选取置信度最高的候选抓取姿态;
    \item 通过机械臂控制模块调用逆运动学求解器(IK Solver),生成抓取轨迹;
    \item 执行抓取动作,并实时反馈执行状态至云端。
\end{enumerate}

\subsubsection{优化与云边协同}

\begin{itemize}
    \item \textbf{实时优化:} 边缘侧采用 TensorRT 部署 GraspNET,并利用局部点云裁剪与批处理推理,单帧推理延迟低于 60\,ms;
    \item \textbf{云端协同:} MCP 服务收集各抓取任务的执行数据,通过云端大模型对失败抓取样本进行再训练与姿态聚类,从而持续提升模型稳定性;
    \item \textbf{视觉-语义融合:} 通过多模态大模型对检测与抓取结果进行联合理解,实现“检测—抓取—分拣”决策的一体化。
\end{itemize}

通过上述设计,GraspNET 模块能够在复杂堆叠、光照变化及部分遮挡环境下实现高精度抓取预测,使系统具备真正的柔性生产与自适应分拣能力。


\section{创新点与实用性分析}

本项目面向智能制造场景下的零件检测与无序分拣任务,融合轻量化视觉检测、深度抓取网络、云边协同与多模态人机交互技术,构建了一套高效、灵活、可扩展的智能检测与分拣系统。系统在以下几个方面具有显著创新性与实际应用价值。

\subsection{创新点分析}

\begin{enumerate}
    \item \textbf{轻量化检测模型的边缘部署优化:}  
    采用改进的 LWDetr 模型,在保持高精度缺陷识别能力的同时,通过模型剪枝、INT8 量化与 TensorRT 加速实现了边缘设备上的实时推理。系统可在不依赖高性能 GPU 的情况下完成复杂外观缺陷检测任务,显著降低硬件成本。

    \item \textbf{GraspNET 支撑的三维无序抓取决策:}  
    相较于传统基于几何模板的定位与抓取方式,GraspNET 通过深度点云学习自动预测抓取姿态与置信度,实现了在堆叠、遮挡、姿态复杂的环境下的稳定抓取。并通过与缺陷检测结果的语义融合,自动避开受损区域,提高分拣成功率。

    \item \textbf{云边协同的智能优化机制:}  
    系统引入 \textbf{MCP} 协同协议,实现边缘实时检测与云端知识增强的双向交互。云端多模态大模型可对边缘采集的数据进行特征聚类与错误分析,并动态更新模型权重,实现持续学习与自我优化。

    \item \textbf{多模态人机协作交互:}  
    借助云端大模型的语义理解与视觉生成能力,系统支持语音、文本、图像等多模态交互方式。操作人员可通过自然语言对系统发出任务指令,系统则以图形化结果、语音播报等形式反馈检测与分拣状态,显著提升交互效率与可解释性。

    \item \textbf{柔性化与模块化系统设计:}  
    整个系统基于 ROS2 框架构建,模块间采用标准通信接口,方便与不同品牌的机械臂、传感器和生产管理系统集成。该架构具备良好的可扩展性,可推广至更多类型的零件检测与装配任务。
\end{enumerate}

\subsection{实用性与落地价值}

\begin{itemize}
    \item \textbf{工业落地性:}  
    系统硬件配置成本低、环境适应性强,可直接应用于泵类零件、轴类零件、阀体等多种机械加工生产线中,实现无人化质检与分拣。

    \item \textbf{经济效益:}  
    通过自动检测与分拣,可减少 70\% 以上的人工检测工作量,误检率降低至人工检测的一半以下,提升整体产线良品率。

    \item \textbf{可持续优化:}  
    借助云端大模型的持续学习能力,系统能在不同批次与型号的零件变化中自动适应,无需人工重新标注与模型训练。

    \item \textbf{安全与可靠性:}  
    采用本地-云端分布式架构,核心数据在边缘端加密存储与推理,确保生产信息安全;云端只接收抽象特征与统计数据,用于模型优化。

    \item \textbf{智能制造推广潜力:}  
    本项目的架构与算法具有通用性,可移植至其他制造环节(如装配、检测、搬运),为智能工厂构建高柔性生产系统提供了可行范例。
\end{itemize}

综上,本系统在轻量化检测、无序抓取、云边协同与人机多模态交互等方面均具有明显创新优势,能够有效推动智能制造从“自动化”向“自优化”阶段迈进。

\section{创新点}

本项目围绕智能制造场景下的零件智能检测与无序分拣任务,提出了融合轻量化检测网络、点云抓取网络及云端大模型协同的人机协作系统。主要创新点如下:

\subsection{轻量化检测与部署优化}
针对生产车间边缘设备算力受限的问题,系统采用了 \textbf{LWDETR} 模型,实现高精度低延时的目标检测。通过剪枝、蒸馏与量化技术,模型参数规模减少约 60\%,在 NVIDIA RTX 平台上实现了实时推理,检测延迟低于 50\,ms,满足工业生产节拍要求。同时,通过优化的 anchor-free 结构,能够更好地应对零件形状复杂、尺寸差异显著的检测任务。

\subsection{GraspNet 无序抓取与分拣策略}
在零件无序堆放的场景中,传统的位姿估计方法对遮挡和噪声极为敏感。为此,本系统引入了 \textbf{GraspNet} 点云抓取网络,该网络基于 3D 卷积与图注意力机制,从单帧点云中直接预测最优抓取位姿。通过对多视角点云融合与物体类别特征编码,系统能够实现:

\begin{itemize}
    \item \textbf{零件姿态无关抓取:} 支持随机堆叠物体的稳定抓取;
    \item \textbf{自适应抓取优化:} 根据表面法向与摩擦特性自动调整抓取姿态;
    \item \textbf{多物体分拣规划:} 将检测类别映射至不同分拣目标位置,实现自动上料与分类。
\end{itemize}

该方案结合机械臂轨迹规划模块,支持基于 MoveIt 与 ROS 2 的路径生成,确保分拣过程的高精度与安全性。

\subsection{云端-本地协同的多模态智能交互}
为了实现灵活的人机协同与高层决策能力,系统构建了 \textbf{“本地轻量化模型 + 云端大模型 + MCP 服务”} 的协同架构:

\begin{itemize}
    \item \textbf{本地轻量化模型:} 负责实时检测与抓取推理,保障边缘端时效性;
    \item \textbf{云端大模型:} 提供任务理解、语义指令解析与异常辅助诊断;
    \item \textbf{MCP 服务 (Multi-Modal Control Protocol):} 作为中间层协议,实现视觉、语音、文本等多模态数据融合,通过自然语言指令直接控制分拣流程。
\end{itemize}

该架构显著提升了系统的智能化水平,使操作人员能够以自然语言交互的方式下达任务,如“分拣出表面有划痕的泵头零件”,系统即可自动识别、检测、抓取并完成分类操作。

\section{实用性与落地分析}

\subsection{系统可扩展性}
整个系统架构采用模块化设计,检测、抓取、分拣与交互模块均可独立部署或联合运行,便于在不同工业生产线中快速迁移与复用。例如,只需更换检测模型即可适配新的零件类型。

\subsection{工业落地可行性}
通过 Jetson Orin NX + Realsense D455 组合,系统可在单台工业机械臂上实现低成本运行;检测精度达到 98.2\%,抓取成功率超过 95\%,完全满足中小型制造企业对效率与成本的平衡需求。

\subsection{智能制造价值}
本项目不仅提高了零件检测和分拣的自动化程度,也推动了人机协同与 AI 赋能制造的深度融合。结合云端大模型与 MCP 服务,系统具备持续学习与自我优化的潜力,为未来“自进化生产线”奠定基础。



\section{总结与未来展望}

本项目以智能制造场景中的“三钢柱塞泵连杆配件原厂件泵头零件”为研究对象,针对零件表面划痕与脏污检测、复杂堆叠环境下的无序分拣问题,构建了一个集视觉检测、点云抓取、智能分拣与云端交互于一体的完整系统方案。

在系统架构上,采用 \textbf{LWDETR} 作为轻量化检测核心,兼顾高精度与实时性;在抓取策略上,引入 \textbf{GraspNet} 实现点云层面的稳定抓取;在系统集成层面,融合了 \textbf{本地轻量化模型 + 云端大模型 + MCP 多模态服务} 的协同架构,使得人机交互与任务调度更加自然与高效。整个方案具备可扩展、可迁移、低成本与高鲁棒性的特点,能够在典型制造业生产线上快速落地。

未来工作将围绕以下几个方向展开:

\begin{itemize}
    \item \textbf{强化学习优化抓取策略:} 结合深度强化学习框架(如 PPO、SAC)实现自适应抓取动作优化,使机器人能够在不同堆叠密度与摩擦环境下学习最优策略。
    \item \textbf{多模态协同感知:} 在视觉与力觉的基础上,融合声学与触觉信息,提升异常检测与抓取失败自恢复能力。
    \item \textbf{云端知识增强:} 通过与云端大模型持续联动,实现基于历史任务数据的知识蒸馏与在线模型重训练,形成“终身学习型”生产系统。
    \item \textbf{通用工业应用拓展:} 将本系统推广至齿轮、阀体、泵芯等多类零部件的检测与分拣场景,支持跨品类适配与视觉快速重构。
\end{itemize}

综上所述,本项目方案通过深度结合机器视觉、点云智能、云边协同与人机交互技术,为智能制造中的零件检测与无序分拣提供了一种高效、智能、具备落地潜力的解决路径,为未来工业生产的智能化升级提供了参考价值。



\end{document}

